\documentclass[conference]{IEEEtran}
\IEEEoverridecommandlockouts
% The preceding line is only needed to identify funding in the first footnote. If that is unneeded, please comment it out.
\usepackage{cite}
\usepackage{amsmath,amssymb,amsfonts}
\usepackage{algorithmic}
\usepackage{graphicx}
\usepackage{textcomp}
\usepackage{xcolor}
\usepackage{hyperref}
\def\BibTeX{{\rm B\kern-.05em{\sc i\kern-.025em b}\kern-.08em
    T\kern-.1667em\lower.7ex\hbox{E}\kern-.125emX}}
%\usepackage{biblatex}
    
%encoding
%--------------------------------------
\usepackage[T1]{fontenc}
\usepackage[utf8]{inputenc}
%--------------------------------------

%Portuguese-specific commands
%--------------------------------------
\usepackage[portuguese]{babel}
%--------------------------------------

%Hyphenation rules
%--------------------------------------
\usepackage{hyphenat}
\hyphenation{mate-mática recu-perar}
%--------------------------------------  
    
\begin{document}

\title{Aprendizagem Automática e COVID-19\\
}

\author{\IEEEauthorblockN{Bruno Carvalho}
\IEEEauthorblockA{\textit{Departamento de Engenharia Informática} \\
\textit{Instituto Superior de Engenharia do Porto}\\
Porto, Portugal \\
1200145@isep.ipp.pt}
\and
\IEEEauthorblockN{Sofia Canelas}
\IEEEauthorblockA{\textit{Departamento de Engenharia Informática} \\
\textit{Instituto Superior de Engenharia do Porto}\\
Porto, Portugal \\
1200185@isep.ipp.pt}
}

\maketitle

\begin{abstract}
Através de dados retirados de uma base de dados internacional, pretende-se prever o impacto da pandemia na população mundial seguindo processos de aprendizagem automática e posterior avaliação dos mesmos.
\end{abstract}

\begin{IEEEkeywords}
análise, dados, COVID-19, pandemia, exploração, inferência, correlação, regressão, classificação, aprendizagem automática, árvores de decisão, redes neuronais, knn, avaliação
\end{IEEEkeywords}

\section{Introdução} %-------------------------------------
No âmbito da pandemia atual, foram extraídos da base de dados internacional “Our World in Data” \cite{database}, dinamizada pela universidade Johns Hopkins University (JHU), dados reais incidentes em 206 países, contendo indicadores acerca da população dos mesmos. 
Pretende-se, recorrendo a processos de aprendizagem automática, prever o impacto da COVID-19 na população mundial, com o objetivo de avaliar os algoritmos quanto à sua aproximação à realidade. Para isso, serão utilizados modelos de regressão e classificação, nomeadamente regressão linear simples e múltipla, árvores de decisão, redes neuronais e knn.
\begin{figure}[htbp]
\centerline{\includegraphics[width=0.95\columnwidth]{images/matrix.png}}
\caption{Matriz de risco}
\label{matrix}
\end{figure}



\section{Metodologia de Trabalho}
\label{methodology} %-------------------------------------
Tendo por base o ficheiro “countryagregatedata.csv”\cite{dataFile}, foi criado um script em R separado em dois tipos de modelos: Regressão e Classificação. Em cada um destes estão presentes alíneas independentes que utilizam algoritmos de aprendizagem automática sobre os dados referentes ao ficheiro. Após a conclusão das diferentes alíneas, foi realizada a comparação e avaliação dos algoritmos, onde a discussão de resultados se encontra nas secções \ref{regression} e \ref{classification} deste artigo. 



\section{Revisão do Estado da Arte} %-------------------------------------
A aprendizagem automática divide-se em três áreas, sendo estas a Supervised Learning, Unsupervised Learning e Reinforcement/Semi-Supervised Learning \cite{algorithms}.
Na área de Supervised Learning os modelos são construídos tendo em conta um processo de treino onde o algoritmo calcula as previsões e recebe o resultado correto, comparando-o, posteriormente, com a previsão obtida. Alguns destes algoritmos são os de regressão e de classificação: Regressão Linear, Modelo KNN, Árvores de Decisão e Redes Neuronais, entre outros \cite{regressionalgorithms}.
Relativamente à Unsupervised Learning, os modelos tentam criar estruturas através dos dados de input, com o objetivo de organizar os dados por semelhança. Alguns dos algoritmos presentes nesta área são do tipo de clustering \cite{clusteringalgorithms} e de Aprendizagem por Regra de Associação \cite{associationrulealgorithms}.
A última área referida reúne os objetivos das áreas referidas anteriormente, ou seja, procura organizar os dados em estruturas por semelhança e também fazer previsões dos mesmos. Os algoritmos utilizados nesta área são extensões dos algoritmos de regressão e classificação, referidos anteriormente.
Para a avaliação dos algoritmos de aprendizagem automática destacam-se: a Matriz de Confusão, que sumariza a performance através dos termos “True Positive”, “True Negative”, “False Positive” e “False Negative”; os valores da Accuracy, Precision, Recall e F1 score, que são calculados através da matriz de confusão; Threshold; AUC-ROC; entre outros. Os algoritmos de regressão contêm medidas próprias para a sua avaliação sendo estas o Erro Absoluto Médio (MAE), Erro Quadrático Médio (MSE), Raiz do Erro Quadrático Médio (RMSE) e o R quadrado \cite{evaluationmetrics}.



\section{Exploração e Preparação dos Dados} %-------------------------------------
No início do script em R estão presentes as importações de bibliotecas necessárias para a realização dos algoritmos, avaliações e testes utilizados. De seguida, encontram-se funções criadas para calcular valores de avaliação dos algoritmos, assim como para visualização dos mesmos na consola.
Como já referido na secção \ref{methodology}, o script está organizado por 2 partes (Regressão e Classificação), cada uma contendo alíneas em que são utilizados algoritmos diferentes e feitas comparações e/ou avaliações. Nestas alíneas são obtidos os dados de treino e teste e a sua posterior incorporação nos algoritmos. Na parte final é feita a avaliação do algoritmo utilizado e, no caso de ser mais do que um, é feita a comparação entre os mesmos.
De forma a igualar a dimensão dos dados de treino e teste perante todos os algoritmos, utilizou-se o critério holdout 70\% treino e 30\% teste em todos os pontos, onde também foram eliminadas as colunas “continent” e “location” uma vez que o algoritmo de redes neuronais não é capaz de processar dados classificados (em texto) e, também, devido à falta de relevância que estes possuem sobre os algoritmos utilizados. Assim, todos os algoritmos utilizam os mesmos dados de teste em cada ponto para uma comparação justa.




\section{Análise e Discussão de Resultados: Regressão}
\label{regression} %-------------------------------------

\subsection{Carregamento do ficheiro e a dimensão e sumário dos dados.} 
\label{ex01}
Após a importação dos dados contidos no ficheiro, é possível verificar a sua dimensão, sendo esta de 209 linhas (cada uma referente a um país) e 25 colunas, referentes a indicadores acerca da população.
\begin{figure}[htbp]
\centerline{\includegraphics[width=0.95\columnwidth]{images/01.png}}
\caption{Sumário dos dados importados}
\label{summary}
\end{figure}
\begin{equation}
y = \frac{y-min_{y}}{max_{y}-min_{y}} \label{norm}
\end{equation}
Através do sumário dos dados (Fig. X) é possível perceber que estes precisarão de ser normalizados para serem incluídos nos algoritmos de redes neuronais e knn, pelo que se procedeu à normalização dos mesmos através da função representada pela equação \eqref{norm}.
Na normalização dos dados foram excluídas as colunas “continent” e “location” por não apresentarem relevância na aprendizagem automática e por não serem dados numéricos. Estes dados foram utilizados ao longo dos pontos do artigo onde a sua inclusão no algoritmo era necessária.


\subsection{Diagrama de correlação entre todos os atributos}
X


\subsection{Regressão linear simples entre “new$\_$cases” e “total$\_$deaths”}

\subsubsection{Função linear resultante}
\begin{equation}
y = 1056.156 + 2.991x\label{3a_equation}
\end{equation}
\begin{equation}
R^{2}_{ajust.} = 0.6835\label{3a_r2ajust}
\end{equation}
\begin{equation}
p-value = 2.2*10^{-16}\label{3a_pvalue}
\end{equation}
X

\subsubsection{Diagrama de dispersão e reta correspondente ao modelo de regressão}
\begin{figure}[htbp]
\centerline{\includegraphics[width=0.95\columnwidth]{images/03.png}}
\caption{Gráfico de dispersão e reta de regressão linear}
\label{3a}
\end{figure}
X

\subsubsection{Erro médio absoluto (MAE) e raiz quadrada do erro médio (RMSE)}
\begin{equation}
MAE = 2051.678\label{3a_r2ajust}
\end{equation}
\begin{equation}
RMSE = 4613.71\label{3a_pvalue}
\end{equation}
X


\subsection{Previsão da esperança de vida aplicando regressão linear múltipla, árvore de regressão e rede neuronal.}
X

\subsubsection{Regressão linear múltipla}
Para a regressão linear múltipla utilizaram-se todos os atributos (excetuando o "continent" e "location", como referido anteriormente) para prever a esperança de vida.

\begin{figure}[htbp]
\centerline{\includegraphics[width=0.95\columnwidth]{images/04_1.png}}
\caption{Resumo do modelo da função de regressão linear múltipla}
\label{4a}
\end{figure}

No sumário da função de regressão obtida, presente na Fig.\ref{4a}, observam-se os coeficientes obtidos e, também, que apenas os parâmetros da “population$\_$density”, “cardiovasc$\_$death$\_$rate”, “female$\_$smokers”, “hospital$\_$beds$\_$ per$\_$thousand”, “human$\_$development$\_$index”, “positive$\_$rate”, “Tot$\_$dead$\_$pop” e “incidence” é que têm relação linear com os valores da esperança de vida, uma vez que os seus p-values são inferiores a 0.05, logo, consideram-se estatisticamente significativos.

\begin{equation}
R^{2}_{ajust.} = 0.8199\label{4a_r2ajust}
\end{equation}
\begin{equation}
p-value = 2.2\times 10^{-16}\label{4a_pvalue}
\end{equation}

Os valores presentes nas equações \eqref{4a_r2ajust} e \eqref{4a_pvalue} permitem concluir, em primeiro lugar, que existe alguma correlação entre os atributos e a esperança de vida, dado que 0.8199 está algo próximo de 1 e, em segundo, que esta é estatisticamente significativa pois o p-value é inferior a 0.05.
Com a obtenção da função de regressão, testou-se a mesma com a previsão dos dados de teste, sendo que os valores do erro absoluto médio (MAE) e a sua raiz quadrada (RMSE) são os apresentados nas equações \eqref{4a_mae} e \eqref{4a_rmse}, respetivamente.

\begin{equation}
MAE=5.416697\label{4a_mae}
\end{equation}
\begin{equation}
RMSE=24.49499\label{4a_rmse}
\end{equation}


\subsubsection{Árvore de regressão}
A árvore de regressão para a variável da esperança de vida foi obtida com a função rpart e com o método ANOVA. O resultado obtido é o apresentado na Fig.\ref{4b}.

\begin{figure}[htbp]
\centerline{\includegraphics[width=0.95\columnwidth]{images/04_2.png}}
\caption{Árvore de regressão para a variável life$\_$expectancy}
\label{4b}
\end{figure}

Com a árvore de regressão construída, realizou-se a previsão e avaliação da mesma, sendo que foram obtidos os valores apresentados do erro médio absoluto (MAE) e a sua raiz quadrada (RMSE).

\begin{equation}
MAE=2.907016\label{4b_mae}
\end{equation}
\begin{equation}
RMSE=3.828154\label{4b_rmse}
\end{equation}


\subsubsection{Redes neuronais}
Através dos dados normalizados foram construídas três redes neuronais com parâmetros diferentes, sendo estes: uma rede com 1 nó interno; outra com 4 nós internos e outra com 2 níveis internos com 5 e 3 nós. Os resultados gráficos e matemáticos de cada rede são apresentados abaixo.

1 nó interno:
\begin{figure}[htbp]
\centerline{\includegraphics[width=0.95\columnwidth]{images/04_3.png}}
\caption{Rede neuronal com 1 nó interno para a variável life$\_$expectancy}
\label{4b}
\end{figure}
\begin{equation}
MAE=0.08414757\label{4c1_mae}
\end{equation}
\begin{equation}
RMSE=0.1493318\label{4c1_rmse}
\end{equation}

4 nós internos:
\begin{figure}[htbp]
\centerline{\includegraphics[width=0.95\columnwidth]{images/04_4.png}}
\caption{Rede neuronal com 4 nós internos para a variável life$\_$expectancy}
\label{4b}
\end{figure}
\begin{equation}
MAE=0.09749222\label{4c2_mae}
\end{equation}
\begin{equation}
RMSE=0.1625733\label{4c2_rmse}
\end{equation}

2 níveis internos com 5 e 3 nós:
\begin{figure}[htbp]
\centerline{\includegraphics[width=0.95\columnwidth]{images/04_4.png}}
\caption{Rede neuronal com 5 e 3 nós internos para a variável life$\_$expectancy}
\label{4b}
\end{figure}
\begin{equation}
MAE=0.09220327\label{4c3_mae}
\end{equation}
\begin{equation}
RMSE=0.2227114\label{4c3_rmse}
\end{equation}


Através dos erros calculados para cada rede neuronal é possível concluir que há uma perda na precisão da previsão com o aumento de níveis e nós internos, já que a melhor rede neuronal desta amostra é aquela com apenas um nó interno. Esta conclusão é retirada através dos RMSEs, onde a primeira rede apresenta um valor inferior às restantes.

Com os resultados obtidos nos três modelos realizados, é possível tirar conclusões referentes à eficiência de cada um deles. O modelo que apresenta um menor erro médio absoluto (MAE) é a rede neuronal com 1 nó interno, que resultou num erro médio muito inferior aos restantes modelos sendo, assim, o melhor modelo destes três.
A regressão linear múltipla apresenta o pior erro médio, ou seja, a árvore de regressão foi o segundo melhor modelo ficando com um erro médio sensivelmente no meio dos valores do melhor e pior modelos.

\subsubsection{Teste aos resultados dos dois melhores modelos}
Por fim realizou-se um teste para comparar as médias dos erros dos dois melhores modelos, sendo estes a Árvore de Regressão e a melhor Rede Neuronal (1 nó interno).

\begin{equation}
Shapiro-Wilk_{p-value}=1.96\times 10^{-11}\label{4_shapiro}
\end{equation}
\begin{equation}
Lillierfors_{p-value}=6.656\times 10^{-13}\label{4_lillie}
\end{equation}
Antes de fazer o teste, verificou-se a normalidade dos dados através de um teste de Shapiro- Wilk e Lillierfors, que resultaram nos p-values apresentados em \eqref{4_shapiro} e \eqref{4_lillie}. Estes valores permitem concluir que os dados não têm distribuição normal pois ambos os valores são inferiores a 0.05.

\begin{equation}
p-value=1.221\times 10^{-5}\label{4_levene}
\end{equation}
Assim, há a implicação da realização de um t.test, já que os dados não apresentam normalidade. Com isto, realizou-se um Levene Test para verificar as igualdades das variâncias, sendo que o resultado deste teste permite concluir que não o são, visto que o p-value é inferior a 0.05 \eqref{4_levene}.

\begin{equation}
  \begin{array}{l}
    H_{0}:\mu _{rpart} - \mu _{neural}=0 \\ 
    H_{1}:\mu _{rpart} - \mu _{neural}\neq 0
  \end{array}\label{4_hypothesis}
\end{equation}
\begin{equation}
p-value=1.214\times 10^{-5}\label{4_ttest}
\end{equation}
O teste foi realizado com as hipóteses referidas em \eqref{4_hypothesis} e tendo em conta a diferença das variâncias verificadas no Levene Test. O resultado obtido permite concluir que há diferenças significativas entre as médias dos erros dos dois melhores modelos, a um nível de significância de 5\%, já que o p-value é inferior a 0.05.



\section{Análise e Discussão de Resultados: Classificação} 
\label{classification} % ------------------------------------------------------

\subsection{Derivação de um novo atributo NiveldeRisco, discretizando o atributo Taxa de Transmissibilidade, em 2 classes: low e high usando como valor de corte a média do atributo.}
\label{ex05}
Com o objetivo de separar os dados da Taxa de Transmissibilidade em duas classes, obteve-se o valor da média dos mesmos (X).
\begin{equation}
\mu = 1.057654\label{4_ttest}
\end{equation}
Através deste valor foi possível fazer a separação dos dados, onde o valor de low ocorre em 75 países e o valor high ocorre em 134 países. Isto permite concluir que a maioria dos países presentes nos dados têm um índice de transmissibilidade superior a 1 e superior à própria média dos países.


\subsection{Avaliação da capacidade preditiva, através do k-fold cross validation, relativamente ao novo atributo NiveldeRisco usando árvore de regressão, rede neuronal e k-vizinhos-mais-próximos.}
\subsubsection{Árvore de decisão}
X


\subsubsection{Rede neuronal}
X


\subsubsection{K-vizinhos-mais-próximos}
X


\subsubsection{k-fold cross validation}
X

\subsubsection{Teste aos resultados dos dois melhores modelos}
X


\subsection{Derivação do novo atributo ClassedeRisco, discretizando os atributos Taxa de Transmissibilidade R(t) e Incidência.}
Para a criação do atributo ClassedeRisco, verificaram-se os valores de Rt e Incidência para atribuir as classes “Verde”, “Amarelo” e “Vermelho” com base na Matriz de Risco Fig.\ref{matrix}.
Após esta classificação, verificaram-se o número de países que estão em cada região, tendo obtido os seguintes valores:
Verde – 55
Amarelo – 34
Vermelho – 120
Mais uma vez, a maioria dos países encontra-se na zona com os piores valores (zona vermelha), tendo o valor do Rt um contributo significativo, como observado nas conclusões do ponto '\ref{ex05}'.



\subsection{Avaliação da capacidade preditiva relativamente ao novo atributo ClassedeRisco usando árvore de regressão, rede neuronal e k-vizinhos-mais-próximos.}
\subsubsection{Árvore de decisão}
A árvore de regressão foi criada de maneira idêntica aos exercícios anteriores com o método “class”, uma vez que este atributo exige uma análise classificativa dos dados. A árvore obtida encontra-se presente na Fig.\ref{8a_rpart}.
\begin{figure}[htbp]
\centerline{\includegraphics[width=0.95\columnwidth]{images/08_1.png}}
\caption{Árvore de decisão para a varíavel ClassedeRisco}
\label{8a_rpart}
\end{figure}
Com o modelo obtido, obtiveram-se os valores de avaliação presentes na Fig.\ref{8a_confusionmatrix}.
\begin{figure}[htbp]
\centerline{\includegraphics[width=0.95\columnwidth]{images/08_2.png}}
\caption{Matriz de confusão e valores de avaliação do modelo da árvore de regressão}
\label{8a_confusionmatrix}
\end{figure}


\subsubsection{Rede neuronal}
Na preparação dos dados para a criação de uma rede neuronal, foi necessário utilizar os dados normalizados no ponto '\ref{ex01}' e também a coluna ClassedeRisco, criada no ponto anterior. Como os dados desta nova coluna são classificados, houve a necessidade de criar colunas extras que continham os valores de “true”/”false” que diferenciavam as classes.
Após esta preparação, foi criada a rede neuronal com 3 nós internos, podendo esta ser observada na Fig.\ref{8b_neural}.
\begin{figure}[htbp]
\centerline{\includegraphics[width=0.95\columnwidth]{images/08_3.png}}
\caption{Rede neuronal com 3 nós internos para a varíavel ClassedeRisco}
\label{8b_neural}
\end{figure}
A Matriz de Confusão e os valores provenientes da mesma da rede neuronal criada estão indicados na Fig.\ref{8b_confusionmatrix}.
\begin{figure}[htbp]
\centerline{\includegraphics[width=0.95\columnwidth]{images/08_4.png}}
\caption{Matriz de confusão e valores de avaliação do modelo da rede neuronal}
\label{8b_confusionmatrix}
\end{figure}


\subsubsection{K-vizinhos-mais-próximos}
X
Os resultados dos valores de avaliação para este modelo encontram-se na Fig.\ref{8c_confusionmatrix}.
\begin{figure}[htbp]
\centerline{\includegraphics[width=0.95\columnwidth]{images/08_5.png}}
\caption{Matriz de confusão e valores de avaliação do modelo Knn}
\label{8c_confusionmatrix}
\end{figure}


\subsubsection{Comparação dos modelos}
X



\section{Conclusões} % ------------------------------------------------------
X

\begin{thebibliography}{00}

\bibitem{database} Ritchie, H. (2021, 31 de maio). \textit{Coronavirus Source Data}. Our World in Data. \url{https://ourworldindata.org/coronavirus-source-data}

\bibitem{dataFile} Our World in Data (2021, 31 de maio). [Ficheiro Csv]

\bibitem{algorithms} Brownlee, J. (2019, 12 de agosto). \textit{A Tour of Machine Learning Algotithms}. Machine Learning Mastery. \url{https://machinelearningmastery.com/a-tour-of-machine-learning-algorithms/}

\bibitem{regressionalgorithms} Ohri, J. (2017, 16 de fevereiro). \textit{Popular Regression Algorithms In Machine Learning Of 2021}. Jigsaw Academy. \url{https://www.jigsawacademy.com/popular-regression-algorithms-ml/}

\bibitem{clusteringalgorithms} McGregor, M, (2020, 21 de setembro). \textit{8 Clustering Algorithms in Machine Learning that All Data Scientists Should Know}. Free Code Camp. \url{https://www.freecodecamp.org/news/8-clustering-algorithms-in-machine-learning-that-all-data-scientists-should-know/ }

\bibitem{associationrulealgorithms} Shaier, S. (2019, 18 de março). \textit{ML Algorithms: One SD - Association Rule Learning Algorithms}. Towards Data Science. \url{https://medium.com/@Shaier/ml-algorithms-one-sd-%CF%83-association-rule-learning-algorithms-b35303e215d}

\bibitem{evaluationmetrics} Mansah. (2020, 24 de novembro). \textit{A Tour of Evaluation Metrics for Machine Learning}. Analytics Vidhya. \url{https://www.analyticsvidhya.com/blog/2020/11/a-tour-of-evaluation-metrics-for-machine-learning/}

\end{thebibliography}

\end{document}
